\documentclass[10pt]{article}
\usepackage{graphicx} % Required for inserting images
\usepackage{geometry}
\usepackage{amsmath}
\usepackage{amsfonts}
\usepackage{physics}

\geometry{
    a4paper,
    left=20mm,
    right=20mm,
    top=20mm,
    bottom=20mm
}

% commands
\newcommand{\deq}{\vcentcolon=}
\newcommand{\idd}{\text{đ}}
\newcommand{\nimplies}{\centernot\implies}
\newcommand{\vc}[1]{\boldsymbol{#1}}
\DeclareMathOperator*{\argmax}{arg\,max}
\DeclareMathOperator*{\argmin}{arg\,min}
\newcommand{\mat}[1]{\mathbf{#1}}
\newcommand{\hess}[1]{\vc{\nabla}^2 f(\vc{#1})}

\title{nla hw05}
\author{cs}

\begin{document}

\maketitle

\section{Implementation Summary}

The provided notebook implements a spatial agent-based simulation on a
$100\times100$ lattice representing the topology of Edinburgh. The map incorporates
hard-coded geographic features, including parks (e.g., The Meadows), roads, and
fixed amenities (schools, gyms, shops). Agents are initialized with income levels
(derived from Scottish Index of Multiple Deprivation data) and heterogeneous
attributes including religion, language, and age category.

The simulation evolves through two coupled dynamic processes:

\begin{enumerate}
    \item \textbf{Price Evolution:} Housing prices $V$ at location $\mathbf{x}$
    are updated iteratively based on occupant income $A$ and the average price
    of the Moore neighborhood $\mathcal{N}(\mathbf{x})$:
    $$ V^{t+1}(\mathbf{x}) = V^t(\mathbf{x}) + A^t(\mathbf{x}) + \lambda
    \frac{\sum_{\mathbf{y}\in\mathcal{N}(\mathbf{x})}
    V^t(\mathbf{y})}{\#\mathcal{N}(\mathbf{x})} $$

    \item \textbf{Agent Mobility:} Unlike the vacancy-based model described in
    the text, the code utilizes pairwise swapping (Kawasaki dynamics). Two
    randomly selected grid cells $\mathbf{x}$ and $\mathbf{y}$ exchange
    occupants if the swap yields a positive net change in utility ($\Delta > 0$).
    The utility function balances economic affordability with social amenities:
    $$ \Delta = \Delta_{\text{money}} + \Delta_{\text{amenities}} $$
    where $\Delta_{\text{money}}$ minimizes the squared difference between
    household income and property price, and $\Delta_{\text{amenities}}$
    maximizes cultural similarity with neighbors (Schelling mechanism) and
    proximity to preferred facilities.
\end{enumerate}

\end{document}
